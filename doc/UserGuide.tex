\documentclass[12pt]{article}
\setlength{\textwidth}{6.5in}
\setlength{\textheight}{9in}
\setlength{\topmargin}{-.5in}
\setlength{\oddsidemargin}{0in}
\setlength{\evensidemargin}{0in}

%\usepackage{graphics}
%\usepackage{amsfonts}
%\usepackage{amssymb}
%\usepackage{epsfig}

%\linespread{1.3}

\begin{document}

\title{User Guide}


\author{
  ACES III Documentation \\
  \\
  \\
  {\bf Victor Lotrich} \\
  {\bf Mark Ponton} \\
  {\bf Norbert Flocke} \\
  {\bf Ajith Perera} \\
  {\bf Erik Deumens} \\
  {\bf Rodney Bartlett} \\
{\em Quantum Theory Project} \\
{\em University of Florida} \\
{\em Gainesville, FL 32605} \\
\\
}
\maketitle

\newpage

\section{\bf Overview} 

ACES III is a program which implements much of the functionality of ACES II in parallel.  
The program is designed to run on a number of Unix-based platforms, including AIX, 
Altix, Cray, and a number of Linux clusters. Although it retains some features of 
ACES II, ACES III is a completely new program, based on the Super Instruction 
Architecture Language (SIAL for short, pronounced "sail") developed by ACES QC in 
conjunction with the DoD High Performance Computing Modernization Program. The program was 
designed to attain excellent performance and scalibility on up to 1000 processors, and
beyond.\\  
\\
Using ACES III, the following types of calculations can be performed, on both closed 
shell and open shell molecular systems:

\begin{itemize} 

 \item SCF, MBPT(2), and CCSD energy and gradient calculations.
 \item CCSD(T) energy calculations.
 \item SCF and MBPT(2) analytic Hessians.

\end{itemize} 

\noindent 
There is also the capability to perform MBPT(2) energy, gradient and Hessian calculations 
with an ROHF reference.\\ 
\\
The serial ACES II contains the capability to do other types of calculations as well. 
(DFT, CCSDT, STEOM for example. For a complete list see the ACES II documentation). 
However, new functionality is being continually added to ACES III 
as the need arises.

\section{\bf Quick Start Guide for Running ACES III:}

\begin{enumerate} 

\item Build a run directory on the file system on which you wish to run the job.  
         Most MPP systems have one or more scratch file systems set up specifically 
         for user jobs.  This is normally the directory on which to set up the run directory.

\item Two files must be present in the run directory before running the ACES III 
         executable.  These are the ZMAT and GENBAS.  The GENBAS contains the data 
         describing basis set information, exactly as in a serial ACES II run.  
         The ZMAT is similar to the ACES II ZMAT and contains the user's input parameters.  
         In addition to the standard ACES II ZMAT, it also may contain a $*$SIP namelist 
         section with  additional parameters specific to ACES III.  These parameters are 
         described in the Parameter Description section below.  If the default 
         ACES III-specific parameter values are acceptable, there is no need to code 
         the $*$SIP section, and the ZMAT is identical to that of a normal ACES II run.

\item A run script should be created to run the job. This script normally contains 
         parameters for the batch queuing system of the computer platform. It also must 
         sets an environmental variable ACES\_EXE\_PATH to the path of the ACES III executable 
         program, xaces3.  Then the script should run the xaces3 executable. Under many 
         systems this involves execution of the mpirun command. 

\end{enumerate} 

\noindent 
{\bf Example:}  

\noindent 
This is an example of a script used on cobalt, an Altix system at NCSA. This example shows 
the following : 

\begin{enumerate} 

\item The PBS parameters set the job name, runtime limit, number of processors, and job 
         queue to use.

\item Some environment variables are set up, including ACES\_EXE\_PATH. They must be 
         exported so that each processor spawned by the mpirun command receives the values. 

\item A run directory is created (\$TMPD), and the ZMAT and GENBAS files are copied 
         into it.  On most systems, it is preferable to also copy the xaces3 file into the 
         run directory for performance reasons, as is done here.

\item Note the mpirun command. It runs xaces3 on 4 processors, using 

         \$TESTROOT/\$atom/\$type/CH4\_AO\_S.out as the stdout file. The "dplace -s1 -c0-3" is 
         specific to Altix systems, telling the system to "pin" the memory to specific 
         processors instead of allowing it to migrate from one processor to another.

\end{enumerate} 

\noindent 
\#!/bin/ksh\\ 
\\
\#PBS -S /bin/ksh\\ 
\#PBS -N CH4\_AO\_S\\ 
\#PBS -j oe\\ 
\#PBS -l ncpus=4\\ 
\#PBS -l walltime=04:00:00\\ 
\#PBS -q standard\\ 
\\
\#\#\#\#\#\#\#\#\#\#\#\#\#\#\#\#\#\#\#\#\#\#\#\\
\#\# PORTING VARIABLES \#\#\\ 
\#\#\#\#\#\#\#\#\#\#\#\#\#\#\#\#\#\#\#\#\#\#\#\\ 
\\
tag=CH4\_AO\_S\\
atom=CH4\\ 
type=AO\_S\\ 
nprocs=4\\ 
out=\$tag.out\\ 
\\ 
\#\#\#\#\#\#\#\#\#\#\#\#\#\#\#\#\#\#\#\#\#\#\#\#\#\#\#\#\\
\#   Set up run environment \#\\
\#\#\#\#\#\#\#\#\#\#\#\#\#\#\#\#\#\#\#\#\#\#\#\#\#\#\#\#\\
\\ 
export WORKDIR=/scratch/users/ponton\\ 
export ACES\_EXE\_PATH=/u/ac/ponton/ACESII/bin\\ 
export TESTROOT=/u/ac/ponton/xaces3\_tests\\ 
\\
export MPI\_REQUEST\_MAX=100000\\ 
\\
TMPD=\$WORKDIR/\$tag\\ 
rm -rf \$TMPD\\ 
mkdir -p \$TMPD\\ 
\\
cd \$TMPD\\ 
cp \$TESTROOT/\$atom/\$type/ZMAT .\\ 
cp \$TESTROOT/GENBAS .\\ 
cp \$ACES\_EXE\_PATH/xaces3 .\\ 
\\ 
mpirun -np \$nprocs dplace -s1 -c0-3 ./xaces3  $>$\$TESTROOT/\$atom/\$type/CH4\_AO\_S.out\\ 
\\ 
Here is the ZMAT file for this job:\\ 
\\
CH4\\ 
H    .431     -.762     -.739\\ 
H   -.467      .762     -.426\\ 
H    .778      .198      .739\\ 
H   -.778     -.688      .587\\ 
C   -.008     -.122      .040\\ 
\\
$*$ACES2\\ 
!restart,symmetry=off\\ 
basis=CC-PVTZ,SPHERICAL=OFF \# basis options\\ 
coordinate=cartesian,cc\_conv=10,scf\_conv=8\\ 
ref=uhf,calc=ccsd\\ 
\\ 
$*$SIP\\ 
COMPANY   = 1 1 3 0\\ 
IOCOMPANY = 2 1 1 0\\ 
MAXMEM    = 900\\ 
SIAL\_PROGRAM = scf\_uhf\_isymm\_diis10.sio\\ 
SIAL\_PROGRAM = tran\_uhf\_ao\_sv1.sio\\ 
SIAL\_PROGRAM = ccsd\_uhf\_ao\_sv1\_diis5.sio\\ 
SIAL\_PROGRAM = lambda\_uhf\_ao\_sv1\_diis5.sio\\ 
SIAL\_PROGRAM = one\_grad\_uhf\_ao\_sv1\_diis5.sio\\ 
SIAL\_PROGRAM = two\_grad\_uhf\_ao\_sv1.sio\\ 
\\
Please note the following: 

\begin{enumerate} 

\item The $*$ACES2 section is a normal setup for a serial ACES II CCSD gradient calculation.
         %The only unusual features are that SYMMETRY is turned off, and we have 
         %"DIRECT=ON,INTEGRALS=GAMESS,FOCK=AO".  This is not required, it is used to 
         %prevent writing out unnecessary large integral files from the serial portion of 
         %ACES II code used internally in ACES III.  

\item In the $*$SIP section, the COMPANY and IOCOMPANY parameters divide the 4 processors 
         into "companies" of 3 worker processes and 1 server process (more about this later).

\item The "MAXMEM=900" forces the program to use 900 Mbytes of memory per processor. 
         Different platforms may allow more or less memory per processor than this.

\item A number of SIAL\_PROGRAM parameters are coded.  These identify a sequence of 
         SIAL programs that will be executed in order to calculate the CCSD gradient.  

\end{enumerate} 

\section{\Large \bf Super Instruction Architecture} 

ACES III was developed using the Super Instruction Architecture. This architecture views 
all data as a set of multi-dimensional blocks, usually with between 10000 and 250000 
floating point numbers per block.  A run-time system, called SIP, was developed to manipulate 
these blocks efficiently.  Also, a high-level programming language, called SIAL, was 
implemented, allowing computational chemists to implement algorithms like SCF and CCSD 
fairly quickly and efficiently as SIAL programs. ACES III runs one or more SIAL programs to 
achieve its desired computational task.  The SIAL programs are coded in the $*$SIP section 
of the ZMAT file as shown above. If the $*$SIP section is not present, ACES III will use 
the appropriate defaults to perform the calculation.\\ 
\\ 
The Super Instruction Architecture divides the set of processors into a master process, and 
a number of worker and server processes.  The master (which is also a worker) performs 
initialization and clean up chores for each SIAL program. The workers perform the actual 
computations (tensor contractions, matrix diagonalizations, etc.).  The servers do nothing 
except store and serve data transferred to them by the workers.  Each server process has a number of 
scratch files, which are created in the run directory.  These files are used to hold data 
until the server receives a request from a worker process for the data.  The scratch files 
may be identified as SCR$*$, and are retained only for the duration of the SIAL program.\\ 
\\ 
Workers and servers are configured by the COMPANY and IOCOMPANY parameters in the $*$SIP 
section of the ZMAT file (described below). Usually, a 3-to-1 ratio of workers to servers 
has been found to achieve good levels of performance.\\
\\ 
A portion of the serial ACES II code, joda, is linked into ACES III, and executed at the 
beginning of each new SIAL program.  The joda code uses data such as the gradient 
calculated within the individual SIAL programs to determine criteria for convergence of 
geometry optimizations and transition state searches, as well as vibrational frequency 
calculations.  If the necessary parameters are set in the $*$ACES2 section to do geometry 
optimization, transition state search, or vibrational frequency calculations, the master 
process continues looping through all the SIAL programs until joda sets flags which indicate 
the job should be halted.  At this time, scratch files are cleaned up and the job is 
terminated by the master.\\ 
\\ 
ACES III has a "coarse grained" restart capability.  Restart is performed at the SIAL 
program level. Each SIAL program passes data to the next SIAL program in the sequence by 
writing the data to a BLOCKDATA file in the run directory.  If a job should terminate due 
to timeout (or some other reason), the run directory contains all necessary information to 
restart the job.  A user can simply modify the run script so as not to disturb any files 
in the run directory (except for removing all the SCR$*$ files), and resubmit the job. 
The master process will restart the job at the beginning of the SIAL program which was in 
progress at the time the previous job died. \\  
\\ 
At the beginning of the job, a dryrun pass is made to estimate if there are enough processors 
to run the job.  Each SIAL program is scanned in the dryrun to determine memory requirements. 
If the dryrun fails, the user should check the printout, which will give an estimate of the 
minimum number of processors required. 

\section{\Large \bf $*$SIP Parameter Description}

The following are the parameters which may be coded in the $*$SIP section. There are no longer 
any required parameters. To override any parameter value, a $*$SIP section must be added to the 
ZMAT file, with a line specifying the parameter to be overridden. Examples are given at the 
end of this document.  

\begin{itemize} 

\item SIP\_MX\_SEGSIZE Blocksize for the atomic orbital dimension of each data block.

\item SIP\_MX\_OCC\_SEGSIZE Blocksize for the occupied orbital dimension 
                        (both alpha and beta spin) of each data block.
\item SIP\_MX\_VIRT\_SEGSIZE Blocksize for the virtual orbital dimension 
                        (both alpha and beta spin) of each data block.
\item MAXMEM   Amount of RAM per process, in Mbytes. Default is 900.
\item COMPANY  Description of the company of worker processes. This consists of 4 
               parameters, the company descriptor, platoon descriptor, number of workers 
               in the company, and memory per worker. Only the number of workers is currently 
               used by the program, the other fields are required, but not used. The default 
               sets the number of workers at 3/4 of the total number of processors.

\item IOCOMPANY Description of the company of I/O server processes. This parameter requires the
                same 4 fields as described under COMPANY above. The default sets the number of 
                I/O servers at 1/4 of the total number of processors.
\item DEBUG\_PRINT Set DEBUG\_PRINT=YES to obtain useful debugging information. Warning: This 
                  could generate a large print file.
\item TIMERS Set TIMERS=YES to obtain extensive timing data for each SIAL program. 
\item AUTO\_SEG\_ALGORITHM This parameter controls the algorithm used
  to generate segment sizes when the parameters SIP\_MX\_SEGSIZE, 
  SIP\_MX\_OCC\_SEGSIZE, and SIP\_MX\_VIRT\_SEGSIZE are not
  coded. The possible values are MEMORY\_OPTIMIZED, which attempts to
  minimize memory usage, and SEGMENT\_OPTIMIZED, which attempts to
  reduce the number of segments into which each coordinate axis is
  sub-divided. The default is SEGMENT\_OPTIMIZED.
\item LOCAL\_PATH The Unix directory path in which to open the server's SCR$*$ files. The default
                 is the directory in which the program is started.
\item MASTER\_IS\_WORKER Obsolete parameter. In an earlier implementation, the master process 
                                           was not required to also be a worker. In the current
                                           implementation, the master is always a process in 
                                           the worker company.
\item INTEGRAL\_PACKAGE Must take on the values ERD or GAMESS.  Although ACES III was 
                       developed with the ability to use either GAMESS or ERD integrals, the 
                       GAMESS integrals do not work for any programs using 2nd derivative 
                       calculations. ACES III now uses only ERD integrals.
\item FAST\_ERD\_MEMCALC This parameter determines whether ERD integral package memory 
                       requirements are estimated by a "fast" algorithm or actually 
                       pre-calculated directly. For large systems using 2nd derivative 
                       integral calculations (i. e. Hessians), the direct calculation can take
                       a significant amount of time. To bypass this calculation and use the 
                       estimation technique, set FAST\_ERD\_MEMCALC = YES.
\item NWORKTHREAD The number of internal memory buffers used by the workers for inter-company 
                  communication. Defaults to 20.
\end{itemize} 
 
\noindent 
The following parameters only apply when using specific SIAL programs. These parameters are 
provided for use in "batch-parallel" jobs, in which the natural parallelism of specific 
types of calculations may be used to break a large problem into a number of smaller ones, and 
the results combined after all the jobs have completed.

\begin{itemize} 
\item IHESS iatom1 iatom2
\item JHESS jatom1 jatom2
   \begin{itemize} 
   \item Used in all Hessian SIAL codes.
   \item The Hessian can be written as a 4-dimensional array hess(iatom,ix,jatom,jx),
         where ix,jx = 1, 3.  IHESS and JHESS represent the range of atoms for which
         the Hessian is computed in the current job.  The default values run the
         entire calculation in one job.

   \item Since correlated Hessian calculations can be quite expensive it is useful
   to allow the user to partition the calculation among several jobs. These
   jobs may be submitted into a batch queue system, and scheduled more
   efficiently by the queuing system.  Instead of running a large Hessian
   calculation over a large number of processors, and having to wait a long
   time for the processors to become available, it may be preferable to use
   IHESS and JHESS to reduce the problem to a number of smaller jobs, each one
   running on a smaller number of processors.

   \item Another reason for dividing the job using IHESS and JHESS is that ACES III
   currently does not allow restarts within a SIAL program.  So, instead of
   running the Hessian calculation as one long-running job, partitioning the
   job in this way allows the calculation to be done in a number of shorter
   jobs.  If one of these small jobs aborts due to a hardware problem, the
   entire amount of time is not lost.

%  \item Note that in order to obtain a correct Hessian the terms with iatom and
%  jatom interchanged must be included in the same job VFL.

   \item In order to use IHESS and JHESS correctly care must be taken to run the
   proper sial codes AND sum the results correctly. Users wishing to
   perform such calculations should contact the authors as this is a nonstandard
   application.
   \end{itemize} 
\item ITRIP ITRIPS ITRIPE
   \begin{itemize} 
   \item ITRIP may be used in any of the SIAL codes for CCSD(T) calculations. It
   is used to partition a large job performing CCSD(T) calculations into
   smaller ones, similar to the way IHESS and JHESS are used to partition
   Hessian calculations.

   \item Since the perturbative triples calculation can be written as
   E = $\sum_i \sum_{abc,jk}$ Z(abc,ijk)$*$T(abc,ijk)
   the contributions E(i) = $\sum_{abc,jk}$ Z(abc,ijk)$*$T(abc,ijk)
   can naturally be computed seperately. ITRIP defines the range of i,
   and has the default ITRIP = 1 max(NOCCA,NOCCB).  This has the effect of
   performing the entire calculation as one job.
   \end{itemize} 
\item SUB SUBB SUBE
   \begin{itemize} 
   \item SUB is used in the CCSD(T) SIAL codes.  It represents the range of
   occupied data to be held within the program's distributed memory.  This
   is used to improve disk performance, and has no effect on the results
   of the calculations.

   \item Ideally SUB and ITRIP coincide. Note that care must be used if the
   ITRIP parameter is used to insure that
   the energy is properly summed. In order to most effectively use
   the SUB parameter the authors should be contacted.
   \end{itemize} 
\end{itemize} 

\section{\Large \bf Example ZMAT Files}
\noindent  
The following are some example ZMAT files, showing how to set up the parameters to run 
different types of ACES III jobs. For a complete list of current SIAL programs, please 
see the SIAL Program Inventory document. 

\subsection{Geometry optimization jobs:} 
%----------------------------------------\\ 
%\\ 
\subsubsection{SCF(UHF) on Ar$_6$ cluster} 
%---------------\\ 
%\\ 
Ar6 IN aug-cc-pvtz basis\\ 
AR  2.5  2.5  0.0\\ 
AR -2.5  2.5  0.0\\ 
AR  2.5 -2.5  0.0\\ 
AR -2.5 -2.5  0.0\\ 
AR  0.0  0.0  2.5\\ 
AR  0.0  0.0 -2.5\\ 
\\ 
$*$ACES2\\ 
!restart,symmetry=off\\ 
GEOM\_OPT=FULL\\ 
basis=AUG-CC-PVTZ,SPHERICAL=ON \# basis options\\ 
UNITS=BOHR\\ 
coordinate=cartesian\\
ref=uhf,calc=scf\\ 
\\ 
$*$SIP\\ 
SIP\_MX\_SEGSIZE      = 30\\ 
SIP\_MX\_OCC\_SEGSIZE  = 27\\ 
SIP\_MX\_VIRT\_SEGSIZE = 27\\
COMPANY   = 1 1 24 0\\
IOCOMPANY = 2 1  8 0\\
MAXMEM    = 900\\
SIAL\_PROGRAM = scf\_uhf\_isymm\_diis10.sio\\ 
SIAL\_PROGRAM = gradscf.sio\\ 
\\

\newpage 

\noindent 
\subsubsection{MP2(UHF) on Ar$_6$ cluster}
%---------------\\ 
%\\ 
Ar6 IN aug-cc-pvtz basis\\ 
AR  2.5  2.5  0.0\\ 
AR -2.5  2.5  0.0\\
AR  2.5 -2.5  0.0\\ 
AR -2.5 -2.5  0.0\\
AR  0.0  0.0  2.5\\
AR  0.0  0.0 -2.5\\
\\
$*$ACES2\\
!restart,symmetry=off\\
GEOM\_OPT=FULL\\
basis=AUG-CC-PVTZ,SPHERICAL=ON \# basis options\\ 
UNITS=BOHR\\
coordinate=cartesian\\ 
ref=uhf,calc=mbpt(2)\\ 
\\ 
$*$SIP\\ 
SIP\_MX\_SEGSIZE      = 30\\ 
SIP\_MX\_OCC\_SEGSIZE  = 27\\
SIP\_MX\_VIRT\_SEGSIZE = 27\\
COMPANY   = 1 1 24 0\\
IOCOMPANY = 2 1  8 0\\
MAXMEM    = 900\\
SIAL\_PROGRAM = scf\_uhf\_isymm\_diis10.sio\\ 
SIAL\_PROGRAM = mp2grad\_uhf\_sv1.sio\\ 
\\

\newpage 

\noindent 
\subsubsection{CCSD(UHF) on Ar$_6$ cluster}
%----------------\\
%\\
Ar6 IN aug-cc-pvtz basis\\ 
AR  2.5  2.5  0.0\\
AR -2.5  2.5  0.0\\
AR  2.5 -2.5  0.0\\
AR -2.5 -2.5  0.0\\
AR  0.0  0.0  2.5\\
AR  0.0  0.0 -2.5\\
\\
$*$ACES2\\ 
!restart,symmetry=off\\ 
GEOM\_OPT=FULL\\ 
basis=AUG-CC-PVTZ,SPHERICAL=ON \# basis options\\ 
UNITS=BOHR\\
coordinate=cartesian\\ 
ref=uhf,calc=ccsd\\ 
\\
$*$SIP\\ 
SIP\_MX\_SEGSIZE      = 30\\ 
SIP\_MX\_OCC\_SEGSIZE  = 27\\
SIP\_MX\_VIRT\_SEGSIZE = 27\\
COMPANY   = 1 1 24 0\\
IOCOMPANY = 2 1  8 0\\
MAXMEM    = 900\\
SIAL\_PROGRAM = scf\_uhf\_isymm\_diis10.sio\\
SIAL\_PROGRAM = tran\_uhf\_ao\_sv1.sio\\
SIAL\_PROGRAM = ccsd\_uhf\_ao\_sv1\_diis5.sio\\
SIAL\_PROGRAM = lambda\_uhf\_ao\_sv1\_diis5.sio\\
SIAL\_PROGRAM = one\_grad\_uhf\_ao\_sv1\_diis5.sio\\
SIAL\_PROGRAM = two\_grad\_uhf\_ao\_sv1.sio\\


\newpage 



\noindent 
\subsection{Transition state search for Dimethylmethylphosphate:}
%--------------------------------\\ 
%\\ 
cccc DMMP NTS1 OPT cccc\\ 
H\\
C 1 r2\\
O 2 r3 1 a3\\
P 3 r4 2 a4 1 d4\\
O 4 r5 3 a5 2 d5\\
C 5 r6 4 a6 3 d6\\
C 4 r7 3 a7 2 d7\\
O 4 r8 3 a8 2 d8\\
H 2 r9 1 a9 3 d9\\
H 2 r10 1 a10 3 d10\\
H 6 r11 5 a11 4 d11\\
H 6 r12 5 a12 4 d12\\
H 6 r13 5 a13 4 d13\\
H 7 r14 4 a14 3 d14\\
H 7 r15 4 a15 3 d15\\
H 7 r16 4 a16 3 d16\\
O 2 r17 4 a17 5 d17\\
H 17 r18 2 a18 4 d18\\ 
\\
r2 =    1.2245104875\\
r3 =    1.4168956473\\
a3 =  109.4716141695\\
r4 =    1.6381450698\\
a4 =  119.9794951257\\
d4 =   62.3758493867\\
r5 =    1.6055092397\\
a5 =  102.2875376988\\
d5 = -139.0505800237\\
r6 =    1.4505339198\\
a6 =  118.5968228514\\
d6 =   73.3164625255\\
r7 =    1.7939415544\\
a7 =  105.1606322982\\
d7 =  115.2130868519\\
r8 =    1.4932730157\\
a8 =  112.3632012565\\
d8 =  -12.8575710693\\
r9 =    1.0866209638\\
a9 =  106.0498560366\\
d9 = -116.2403495711\\
r10=    1.0897248583\\
a10=  106.2524931192\\
d10=  122.5676785489\\
r11=    1.0892520412\\
a11=  109.9657836383\\
d11=   62.4962398535\\
r12=    1.0862503872\\
a12=  105.6295098879\\
d12= -178.3609771686\\
r13=    1.0898070256\\
a13=  110.1634587241\\
d13=  -59.2262253696\\
r14=    1.0896659037\\
a14=  109.1681841289\\
d14=  173.1829774789\\
r15=    1.0901180999\\
a15=  108.8704671880\\
d15=  -67.6638946514\\
r16=    1.0892878937\\
a16=  110.0145490901\\
d16=   52.6755615836\\
r17=    2.4893506911\\
a17=   86.0854081748\\
d17=  -76.7143005164\\
r18=    0.9793571499\\
a18=   88.9637963996\\
d18=  -33.3244903596\\ 
\\
$*$ACES2\\ 
!restart,symmetry=off\\ 
basis=6-31++G$*$$*$,mult=2,spherical=on\\
ref=uhf,calc=ccsd\\
METHOD=TS\\ 
GEOM\_OPT=FULL\\ 
\\
$*$SIP\\ 
SIP\_MX\_SEGSIZE      = 30\\
SIP\_MX\_OCC\_SEGSIZE  = 27\\
SIP\_MX\_VIRT\_SEGSIZE = 27\\
COMPANY   = 1 1 24 0\\
IOCOMPANY = 2 1  8 0\\
MAXMEM    = 900\\
SIAL\_PROGRAM = scf\_uhf\_isymm\_diis10.sio\\ 
SIAL\_PROGRAM = tran\_uhf\_ao\_sv1.sio\\ 
SIAL\_PROGRAM = ccsd\_uhf\_ao\_sv1\_diis5.sio\\ 
SIAL\_PROGRAM = lambda\_uhf\_ao\_sv1\_diis5.sio\\ 
SIAL\_PROGRAM = one\_grad\_uhf\_ao\_sv1\_diis5.sio\\ 
SIAL\_PROGRAM = two\_grad\_uhf\_ao\_sv1.sio\\ 
 

\newpage 

\noindent 
\subsection{Vibrational frequency calculation for the water ion:}
%-----------------------------------------------\\ 
%\\
H2O(-1) in CC-PVQZ basis\\ 
O  0.0  0.0      0.1173\\
H  0.1  0.7572  -0.4692\\
H  0.0 -0.7572  -0.4692\\
\\
$*$ACES2(CALC=MBPT(2),BASIS=CC-PVQZ,REF=UHF,SPHERICAL=ON\\ 
CC\_CONV=8,SCF\_CONV=8\\
VIB\_FINDIF=EXACT\\
charge=1,multiplicity=2\\ 
COORDINATES=CARTESIAN\\
SYMMETRY=OFF)\\
\\
$*$SIP\\
MAXMEM= 900\\
COMPANY   = 1 1 3 0\\
IOCOMPANY = 2 1 1 0\\
SIP\_MX\_SEGSIZE      = 29\\
SIP\_MX\_OCC\_SEGSIZE  = 5\\
SIP\_MX\_VIRT\_SEGSIZE = 28\\
SIAL\_PROGRAM = scf\_uhf\_init.sio\\
SIAL\_PROGRAM = scf\_uhf\_finish.sio\\
SIAL\_PROGRAM = hess\_uhf\_mp2\_seg.sio\\



\newpage 

\noindent 
\subsection{Hessian:}
%-----------------\\ 
%\\
\subsubsection{SCF(UHF) for water ion}
%---------------\\

H2O(-1) in CC-PVQZ basis\\ 
O  0.0  0.0      0.1173\\
H  0.1  0.7572  -0.4692\\
H  0.0 -0.7572  -0.4692\\
\\
*ACES2(CALC=SCF,BASIS=CC-PVQZ,REF=UHF,SPHERICAL=ON\\
CC\_CONV=8,SCF\_CONV=8\\
charge=1,multiplicity=2\\
COORDINATES=CARTESIAN\\
SYMMETRY=OFF)\\ 
\\
$*$SIP\\
MAXMEM= 900\\
COMPANY   = 1 1 3 0\\
IOCOMPANY = 2 1 1 0\\
SIP\_MX\_SEGSIZE      = 29\\ 
SIP\_MX\_OCC\_SEGSIZE  = 5\\
SIP\_MX\_VIRT\_SEGSIZE = 28\\
SIAL\_PROGRAM = scf\_uhf\_init.sio\\ 
SIAL\_PROGRAM = scf\_uhf\_finish.sio\\
SIAL\_PROGRAM = hess\_uhf\_scf.sio\\


\newpage 



\noindent 
\subsubsection{MP2(UHF)}
%----------------\\ 
%\\
H2O(-1) in CC-PVQZ basis\\ 
O  0.0  0.0      0.1173\\ 
H  0.1  0.7572  -0.4692\\ 
H  0.0 -0.7572  -0.4692\\ 
\\
$*$ACES2(CALC=MBPT(2),BASIS=CC-PVQZ,REF=UHF,SPHERICAL=ON\\ 
CC\_CONV=8,SCF\_CONV=8\\ 
charge=1,multiplicity=2\\ 
COORDINATES=CARTESIAN\\ 
SYMMETRY=OFF)\\ 
\\
$*$SIP\\ 
MAXMEM= 900\\ 
COMPANY   = 1 1 3 0\\
IOCOMPANY = 2 1 1 0\\
SIP\_MX\_SEGSIZE      = 29\\
SIP\_MX\_OCC\_SEGSIZE  = 5\\
SIP\_MX\_VIRT\_SEGSIZE = 28\\
SIAL\_PROGRAM = scf\_uhf\_init.sio\\
SIAL\_PROGRAM = scf\_uhf\_finish.sio\\
SIAL\_PROGRAM = hess\_uhf\_mp2\_seg.sio\\


\newpage 


\noindent 
\subsection{Single point energy CCSD(T) calculation for the Ar$_6$ cluster:}
%--------------------------------------------------------\\
%\\
\subsubsection{RHF}
%---------\\
%\\
Ar6 IN aug-cc-pvtz basis\\
AR  2.5  2.5  0.0\\
AR -2.5  2.5  0.0\\
AR  2.5 -2.5  0.0\\
AR -2.5 -2.5  0.0\\
AR  0.0  0.0  2.5\\
AR  0.0  0.0 -2.5\\
\\
$*$ACES2\\
!restart,symmetry=off\\ 
basis=AUG-CC-PVTZ,SPHERICAL=ON \# basis options\\
UNITS=BOHR\\
coordinate=cartesian\\
ref=rhf,calc=ccsd(t)\\
\\
$*$SIP\\
SIP\_MX\_SEGSIZE      = 30\\
SIP\_MX\_OCC\_SEGSIZE  = 27\\
SIP\_MX\_VIRT\_SEGSIZE = 27\\ 
COMPANY   = 1 1 96 0\\
IOCOMPANY = 2 1 32 0\\
MAXMEM    = 900\\
SIAL\_PROGRAM = scf\_rhf\_isymm\_diis10.sio\\
SIAL\_PROGRAM = tran\_rhf\_ao\_sv1.sio\\
SIAL\_PROGRAM = ccsd\_rhf\_ao\_sv1\_diis5.sio\\
SIAL\_PROGRAM = ccsdpt\_rhf\_pp.sio\\


\newpage 


\noindent 
\subsubsection{UHF}
%---------\\ 
%\\ 
Ar6 IN aug-cc-pvtz basis\\ 
AR  2.5  2.5  0.0\\
AR -2.5  2.5  0.0\\
AR  2.5 -2.5  0.0\\
AR -2.5 -2.5  0.0\\
AR  0.0  0.0  2.5\\
AR  0.0  0.0 -2.5\\
\\
$*$ACES2\\ 
!restart,symmetry=off\\ 
basis=AUG-CC-PVTZ,SPHERICAL=ON \# basis options\\ 
UNITS=BOHR\\ 
coordinate=cartesian\\ 
ref=uhf,calc=ccsd(t)\\
\\
$*$SIP\\
SIP\_MX\_SEGSIZE      = 30\\
SIP\_MX\_OCC\_SEGSIZE  = 27\\
SIP\_MX\_VIRT\_SEGSIZE = 27\\
COMPANY   = 1 1 96 0\\
IOCOMPANY = 2 1 32 0\\ 
MAXMEM    = 900\\ 
SIAL\_PROGRAM = scf\_uhf\_isymm\_diis10.sio\\
SIAL\_PROGRAM = tran\_uhf\_ao\_sv1.sio\\
SIAL\_PROGRAM = ccsd\_uhf\_ao\_sv1\_diis5.sio\\ 
SIAL\_PROGRAM = ccsdpt\_uhf\_pp.sio\\ 


\newpage 

\noindent 
\subsection{Single point CCSD gradient(UHF) using DROPMO on the CH$_4$ molecule:}
%--------------------------------------------------------------------------\\ 
%\\
CH4\\ 
H    .431     -.762     -.739\\
H   -.467      .762     -.426\\
H    .778      .198      .739\\
H   -.778     -.688      .587\\
C   -.008     -.122      .040\\
\\
$*$ACES2\\
!restart,symmetry=off\\
dropmo=1-2/30-35\\ 
basis=CC-PVDZ,SPHERICAL=OFF \# basis options\\ 
coordinate=cartesian,cc\_conv=8,scf\_conv=8\\ 
FOCK=AO\\
ref=uhf,calc=ccsd\\ 
\\
$*$SIP\\ 
SIP\_MX\_SEGSIZE      = 20\\
SIP\_MX\_OCC\_SEGSIZE  = 13\\
SIP\_MX\_VIRT\_SEGSIZE = 20\\
COMPANY   = 1 1 3 0\\
IOCOMPANY = 2 1 1 0\\
MAXMEM    = 900\\
SIAL\_PROGRAM = scf\_uhf\_isymm\_diis10.sio\\
SIAL\_PROGRAM = tran\_uhf\_ao\_sv1.sio\\
SIAL\_PROGRAM = ccsd\_uhf\_dropmo.sio\\ 
SIAL\_PROGRAM = lambda\_uhf\_dropmo.sio\\
SIAL\_PROGRAM = expand\_cc.sio\\ 
SIAL\_PROGRAM = tran\_uhf\_expanded.sio\\ 
SIAL\_PROGRAM = one\_grad\_uhf\_ao\_sv1\_dropmo\_diis5.sio\\ 
SIAL\_PROGRAM = two\_grad\_uhf\_ao\_sv1\_dropmo.sio\\ 

\noindent 
Any of the previous examples should also work if the $*$SIP sections are ommitted. 
For example, example 5.6 could be rewritten as follows:\\ 
\\
\\
CH4\\
H    .431     -.762     -.739\\ 
H   -.467      .762     -.426\\ 
H    .778      .198      .739\\ 
H   -.778     -.688      .587\\ 
C   -.008     -.122      .040\\ 
\\
$*$ACES2\\ 
!restart,symmetry=off\\ 
dropmo=1-2/30-35\\ 
basis=CC-PVDZ,SPHERICAL=OFF \# basis options\\ 
coordinate=cartesian,cc\_conv=8,scf\_conv=8\\ 
ref=uhf,calc=ccsd\\ 
\\
In this case, the program determines the segmentation parameters, the worker/server 
configuration, and even which .sio files to run, all based on the REF, CALC, and DROPMO 
parameters. Caution: Some type of calculations performed by the ACES II serial code are not 
yet supported in ACES III. An example of this is ECP. If the program cannot determine the 
type of calculation from the $*$ACES2 parameters, and no $*$SIP section is provided, an 
error message is printed and the program will abort.\\ 
\\
If a user desires to run a different program than that determined by default, a $*$SIP 
section must be provided, and all .sio files must be specified, not just the one that is 
different from the default. For example, suppose someone wished to run a new integral 
transformation program, called test\_tran.sio. It might seem that you could rewrite the 
previous example as follows:\\ 
\\ 
CH4\\ 
H    .431     -.762     -.739\\ 
H   -.467      .762     -.426\\ 
H    .778      .198      .739\\ 
H   -.778     -.688      .587\\
C   -.008     -.122      .040\\
\\
$*$ACES2\\
!restart,symmetry=off\\ 
dropmo=1-2/30-35\\
basis=CC-PVDZ,SPHERICAL=OFF \# basis options\\ 
coordinate=cartesian,cc\_conv=8,scf\_conv=8\\ 
ref=uhf,calc=ccsd\\ 
\\
$*$SIP\\ 
SIAL\_PROGRAM = test\_tran.sio\\ 
\\
However, there is no way to determine where in the sequence of .sio programs that 
test\_tran.sio should be placed. All SIAL\_PROGRAM parameters must be specified to correctly 
override the defaults. The correct ZMAT coding is as follows:\\ 
\\ 
CH4\\ 
H    .431     -.762     -.739\\ 
H   -.467      .762     -.426\\
H    .778      .198      .739\\ 
H   -.778     -.688      .587\\
C   -.008     -.122      .040\\ 
\\
$*$ACES2\\ 
!restart,symmetry=off\\ 
dropmo=1-2/30-35\\ 
basis=CC-PVDZ,SPHERICAL=OFF \# basis options\\ 
coordinate=cartesian,cc\_conv=8,scf\_conv=8\\
ref=uhf,calc=ccsd\\ 
\\
$*$SIP\\  
SIAL\_PROGRAM = scf\_uhf\_isymm\_diis10.sio\\
SIAL\_PROGRAM = test\_tran.sio\\
SIAL\_PROGRAM = ccsd\_uhf\_dropmo.sio\\
SIAL\_PROGRAM = lambda\_uhf\_dropmo.sio\\
SIAL\_PROGRAM = expand\_cc.sio\\
SIAL\_PROGRAM = tran\_uhf\_expanded.sio\\
SIAL\_PROGRAM = one\_grad\_uhf\_ao\_sv1\_dropmo\_diis5.sio\\
SIAL\_PROGRAM = two\_grad\_uhf\_ao\_sv1\_dropmo.sio\\


\end{document}

